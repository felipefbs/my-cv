\cvsection{Experiências}

\begin{cventries}
  \cventry
  {Theia - Empresa nacional do segmento de Healthtech} % Organization
  {Desenvolvedor Back-end Pleno} % Job title
  {São Paulo, SP (Remoto)} % Location
  {Jun, 2022 - Atual} % Date(s)
  {
    \begin{cvitems}
      \item {\textbf{Desenvolvi} e \textbf{implementei} novos serviços em  {Go}, criando funcionalidades como autenticação, upload de arquivos ( {AWS S3}) e sistemas de auditoria ( {AWS SQS}).}
      \item {\textbf{Renovei} a arquitetura dos sistemas para melhor adequação ao trabalho em equipes pequenas, \textbf{otimizando o fluxo de desenvolvimento}.}
      \item {\textbf{Projetei} e \textbf{ampliei} o uso de um pacote comum entre sistemas, consolidando uma biblioteca externa para maior \textbf{eficiência}.}
      \item {\textbf{Liderei} o desenvolvimento da interface entre front-end e back-end, aplicando as \textbf{melhores práticas} para experiência do usuário e desenvolvedores.}
      \item {\textbf{Implementei} ferramentas de rastreabilidade e métricas utilizando {Prometheus} e {Grafana}, aumentando a visibilidade e o monitoramento dos sistemas.}
      \item {\textbf{Aumentei} em 50\% a cobertura de testes dos repositórios principais, utilizando ferramentas como {Testcontainers} para serviços como {AWS S3}, {SQS}, {DynamoDB} e {PostgreSQL}.}
      \item {Linguagens: \textbf{GoLang}, \textbf{SQL}, \textbf{Kotlin}.}
      \item {Tecnologias: \textbf{AWS (S3, ECR, DynamoDB, SQS)}, \textbf{PostgreSQL}, \textbf{Chi}, \textbf{Testcontainers}, \textbf{Github}, \textbf{Swagger}, \textbf{JWT}, \textbf{OAuth 2.0}, \textbf{Prometheus}, \textbf{Grafana}.}
    \end{cvitems}
  }

  \cventry
  {Laboratório Assert - Laboratório de pesquisa e desenvolvimento cientifico} % Organization/Project
  {Desenvolvedor Back-end} % Job title
  {João Pessoa, PB (Remoto)} % Location
  {Mar, 2024 - Mar, 2025} % Date(s)
  {
    \begin{cvitems}
      \item {\textbf{Desenvolvi APIs REST utilizando} {Django Rest Framework} para o gerenciamento de dispositivos  {IoT} e o registro seguro de seus dados, aplicando  {criptografia} para proteger a comunicação entre dispositivos e back-end.}
      \item {\textbf{Planejei} e \textbf{implementei} a \textbf{arquitetura de comunicação} entre dispositivos e sistemas, garantindo alta eficiência e segurança nos fluxos de dados.}
      \item {\textbf{Ampliei} e \textbf{customizei} a API padrão de controle de usuários do {Django}, adaptando-a às necessidades específicas do projeto.}
      \item {\textbf{Implementei} um sistema web que exibe, em tempo real, os dados enviados pelos dispositivos, incluindo \textbf{alertas} configurados com base na severidade das alterações de dados.}
      \item {\textbf{Documentei} os endpoints do sistema utilizando  {OpenAPI (Swagger)}, melhorando a \textbf{rastreabilidade} e a integração com outros desenvolvedores.}
      \item {\textbf{Adotei} as melhores práticas de desenvolvimento seguindo os padrões da documentação oficial e da comunidade \textbf{Django}.}
      \item {Tecnologias: \textbf{Python}, \textbf{Django Rest Framework}, \textbf{PostgreSQL}, \textbf{Swagger (OpenAPI)}, \textbf{Criptografia}, \textbf{WebSockets}.}
    \end{cvitems}
  }


  % \cventry
  % {Apoio Ecolimp} % Organization
  % {Desenvolvedor Back-end} % Job title
  % {Rio de Janeiro, RJ (Remoto)} % Location
  % {Fev, 2022 - Jun, 2022} % Date(s)
  % {
  %   \begin{cvitems} % Description(s) of tasks/responsibilities
  %     \item {Implementação serviços utilizando GoLang em conjunto a diversos padrões de mercado para garantir rastreabilidade, consistência e manutenibilidade de código}
  %     \item {Ampliação e manutenção de diversos serviços em Typescript utilizados para o gerenciamento de limpeza de salas de hospitais}
  %     \item {Linguagens: GoLang, Typescript, MongoDB}
  %     \item {Tecnologias: NestJS, Swagger, AWS}
  %   \end{cvitems}
  % }

  \cventry
  {Laboratório de Sistemas Distribuídos (LSD) - Laboratório de pesquisa e desenvolvimento cientifico} % Organization
  {Desenvolvedor e Pesquisador} % Job title
  {Campina Grande, PB (Remoto)} % Location
  {Ago, 2021 - Jun, 2022} % Date(s)
  {
    \begin{cvitems}
      \item {\textbf{Conduzi pesquisas} em soluções de \textbf{segurança da informação}, desenvolvendo tecnologias baseadas na virtualização do chip  {TPM 2.0} para a plataforma  {VMware}.}
      \item {\textbf{Projetei} e \textbf{implementei} soluções com {vTPM} para aumentar a integridade e segurança de máquinas virtuais, integrando o {VMware SDK} com {GoLang}.}
      \item {\textbf{Contribuí} para a \textbf{robustez e confiabilidade} da infraestrutura virtual, aprimorando a proteção contra ataques e garantindo maior conformidade com padrões de segurança.}
      \item {Linguagens: \textbf{GoLang}, \textbf{Shell Script}.}
      \item {Tecnologias: \textbf{VMware}, \textbf{TPM 2.0}, \textbf{GNU/Linux}.}
    \end{cvitems}
  }

  % \cventry
  % {Núcleo de Tecnologias Estratégicas Em Saúde (NUTES) - UEPB} % Organization
  % {Desenvolvedor Back-end} % Job title
  % {Campina Grande, PB (Remoto)} % Location
  % {Jun, 2021 - Ago, 2021} % Date(s)
  % {
  %   \begin{cvitems} % Description(s) of tasks/responsibilities
  %     \item {Desenvolvimento de soluções para a coleta de dados, garantindo um fluxo contínuo e seguro de informações essenciais}
  %     \item {Implementei, junto com meus colegas de trabalho, o back end da aplicação Sênior Saúde Móvel utilizando arquitetura de microsserviços}
  %     \item {Linguagens: Python, MongoDB}
  %     \item {Tecnologias: Flask, GNU/Linux}
  %   \end{cvitems}
  % }

  \cventry
  {Resilia Educação - Empresa nacional do segmento de EduTech} % Organization
  {Facilitador Tech} % Job title
  {Rio de Janeiro, RJ (Remoto)} % Location
  {Mai, 2021 - Dez, 2021} % Date(s)
  {
    \begin{cvitems}
      \item {\textbf{Facilitei} o aprendizado de \textbf{desenvolvimento web}, auxiliando estudantes a entender conceitos chave e aplicá-los em projetos práticos.}
      \item {\textbf{Ministrei} aulas de desenvolvimento back-end utilizando {Express.js} e {Javascript}, \textbf{orientando estudantes} na criação de APIs.}
      \item {\textbf{Orientei} a modelagem de bancos de dados seguindo boas práticas para manipulação e organização de dados em {MySQL}.}
      \item {\textbf{Instrui} sobre gerenciamento eficiente de containers {Docker}, simplificando fluxos de trabalho para desenvolvedores iniciantes.}
      \item {Linguagens: \textbf{Javascript}, \textbf{SQL}.}
      \item {Tecnologias: \textbf{Docker}, \textbf{Express.js}, \textbf{Node.js}, \textbf{MySQL}.}
    \end{cvitems}
  }

  \cventry
  {Laboratório GCOMPI - Laboratório de pesquisa e desenvolvimento cientifico} % Organization
  {Discente Bolsista} % Job title
  {Campina Grande, PB} % Location
  {Jan, 2020 - Mai, 2021} % Date(s)
  {
    \begin{cvitems}
      \item {\textbf{Realizei} pesquisas em Redes de Sensores Sem Fio, explorando novos métodos de comunicação e coleta de dados.}
      \item {\textbf{Desenvolvi} firmware para a plataforma {OpenMote B}, ampliando a comunicação via rádio com suporte a várias modulações do protocolo {IEEE 802.15.4g}.}
      \item {\textbf{Analisei} dados coletados utilizando {Python}, contribuindo diretamente para a elaboração do Trabalho de Conclusão de Curso.}
      \item {Linguagens: \textbf{C++}, \textbf{Python}, \textbf{SQL-like}.}
      \item {Tecnologias: \textbf{InfluxDB}, \textbf{SCons}, \textbf{FreeRTOS}, \textbf{GNU/Linux}.}
    \end{cvitems}
  }
\end{cventries}
